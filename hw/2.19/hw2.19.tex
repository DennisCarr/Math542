\documentclass[12pt]{article}
\usepackage{amsmath,amsfonts,amsthm}
\usepackage[fleqn]{mathtools}
\usepackage{a4wide}
\usepackage{fancyhdr} % Custom headers and footers
\usepackage{graphicx}
\usepackage{listings}
\usepackage{parskip}

\usepackage[adobe-utopia]{mathdesign}
\usepackage[T1]{fontenc}

\usepackage{titlesec} % www.ctan.org/tex-archive/macros/latex/contrib/titlesec/titlesec.pdf

% Below "\section" can be replaced with "\subsection" and "\subsubsection"
% in order to customize the corresponding headings



\pagestyle{fancyplain}
\fancyhead[L]{HW9: 02/19/2014}
\fancyhead[C]{University of Wisconsin-Madison}
\fancyhead[R]{Professor Arnold Miller}
\fancyfoot[L]{Taylor Lee}
\fancyfoot[C]{MATH 542}
\fancyfoot[R]{tdlee2@wisc.edu}
\newtheoremstyle{moo}
{10pt}
{10pt}
{}
{}
{\bfseries}
{:}
{\newline }
{}


\theoremstyle{moo}
\newtheorem*{prob}{Problem}
\newtheorem*{sol}{Solution}
\newtheorem*{theorem}{Theorem}

\def\zz{{\mathbb Z}}



\begin{document}
\fontencoding {T1}
\fontfamily {put}
\fontseries {\seriesdefault}
\fontshape {\shapedefault}
%\fontsize {size} {baselineskip} 
\selectfont

\title{\usefont{T1}{put}{b}{n} Math 542-Modern Algebra II}
\date{Febuary 19, 2014}         % used by \maketitle
\author{Taylor Lee}      % used by \maketitle                    % used by \maketitle
\maketitle                      % automatic title!



\begin{prob}(Wed Feb 12)
\par (a) Prove that there are no simple groups of order
either $575$ or $272$.
\par (b) For any prime $p$ prove there are no simple groups
of order $p(p-1)$ or $p(p+2)$.
\end{prob}


\begin{sol}
(a)
$272$ factors as $2^4 \times 17$. It is clear that the $17$-Sylow subgroup is normal, since the number of $17$-Sylow subgroups must divide $m = 16$, and also be congruent to one $\pmod{17}$. It is clear, that for $1,2,4,8,16$, the only number which is congruent to one $\pmod{17}$ is $1$, hence there is one $17$-Sylow suggroup, and hence it is normal, making it impossible for a group of order $272$ to be simple.

$575$ factors as $5^2 \times 23$. It is clear that the $23$-Sylow subgroup is normal, since the number of $23$-Sylow subgroups must divide $m = 25$, and also be congruent to one $\pmod{23}$. It is clear, that for $1,5,25$, the only number which is congruent to one $\pmod{23}$ is $1$. Hence, there is only one $23$-Sylow subgroup, and hence it is normal, making it impossible for a group of order $272$ to be simple.

(b) For any prime $p$, a group with order $p(p-1)$ must not be simple, since the number of $p$-Sylow subgroups must divide $p-1$, and also be congruent to $1 \pmod{p}$. Hence, since $1+p$ is the next number above $1$ which is $1 \pmod{p}$ and yet is larger than $p-1$, the only possible number of $p$-Sylow subgroups is $1$, as all other numbers $1 \pmod{p}$ will also be too large to divide $p-1$.

A group of order $p(p+2)$ must not be simple, since the number of $p$-Sylow subgroups must divide $p+2$, and also be congruent to $1 \pmod{p}$. Hence, since $1+p$ is the next number above $1$ which is congruent to $1 \pmod{p}$ and yet is too large to divide $p+2$, the only possible number of $p$-Sylow subgroups is $1$, as all other numbers $1 \pmod{p}$ will also be too large to divide $p+1$. A special case is when $p=2$, and the order of the group is $8$. In this case, the $2$-Sylow subgroup is the group itself, and since $8$ is a power of $2$, this is the only possible $p$-Sylow subgroup and hence is any groups of order $8$ are simple.
\end{sol}








\end{document}   
