\documentclass[12pt]{article}
\usepackage{amsmath,amsfonts,amsthm}
\usepackage[fleqn]{mathtools}
\usepackage{a4wide}
\usepackage{fancyhdr} % Custom headers and footers
\usepackage{graphicx}
\usepackage{listings}
\usepackage{parskip}

\usepackage[adobe-utopia]{mathdesign}
\usepackage[T1]{fontenc}

\usepackage{titlesec} % www.ctan.org/tex-archive/macros/latex/contrib/titlesec/titlesec.pdf

% Below "\section" can be replaced with "\subsection" and "\subsubsection"
% in order to customize the corresponding headings



\pagestyle{fancyplain}
\fancyhead[L]{HW8: 02/17/2014}
\fancyhead[C]{University of Wisconsin-Madison}
\fancyhead[R]{Professor A. Miller}
\fancyfoot[L]{Taylor Lee}
\fancyfoot[C]{MATH 542}
\fancyfoot[R]{tdlee2@wisc.edu}
\newtheoremstyle{moo}
{10pt}
{10pt}
{}
{}
{\bfseries}
{:}
{\newline }
{}


\theoremstyle{moo}
\newtheorem*{prob}{Problem}
\newtheorem*{sol}{Solution}
\newtheorem*{theorem}{Theorem}

\def\zz{{\mathbb Z}}



\begin{document}
\fontencoding {T1}
\fontfamily {put}
\fontseries {\seriesdefault}
\fontshape {\shapedefault}
%\fontsize {size} {baselineskip} 
\selectfont

\title{\usefont{T1}{put}{b}{n} Math 542-Modern Algebra II}
\date{Febuary 17, 2014}         % used by \maketitle
\author{Taylor Lee}      % used by \maketitle                    % used by \maketitle
\maketitle                      % automatic title!


\begin{prob}
(Mon Feb 10) Prove for any $n\geq 3$ that
$Z(S_n)=\{id\}$.
\end{prob}

\begin{sol}
Let $\alpha \in S_n$ be choosen arbitrary such that $\alpha \neq e$ and set $a,b$ such that $\alpha(a) = b$, where $a \neq b$. Then, let $\beta \in S_n$ such that $\beta$ is the two cycle: $\beta = (bc)$, with $c \neq a$. We can find such a $c$ since $n \geq 3$, and so $\beta $ fixes $a$. Now, we can see that:

\[
\beta \alpha \beta^{-1}(a) = \beta \alpha(a) = \beta (b) = c.
\]

Wheras:

\[
\alpha(a) = b.
\]

Hence, $\beta \alpha \beta^{-1} \neq \alpha$, which shows that $\beta \alpha \neq \alpha \beta$, and hence no element in $S_n$ commutes with every other element of $S_n$, other than $e \in S_n$. Hence, $Z(S_n)=e.$
\end{sol}








\end{document}   
