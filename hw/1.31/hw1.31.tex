\documentclass[12pt]{article}
\usepackage{amsmath,amsfonts,amsthm}
\usepackage[fleqn]{mathtools}
\usepackage{a4wide}
\usepackage{fancyhdr} % Custom headers and footers
\usepackage{graphicx}
\usepackage{listings}
\usepackage{parskip}

\usepackage[adobe-utopia]{mathdesign}
\usepackage[T1]{fontenc}

\usepackage{titlesec} % www.ctan.org/tex-archive/macros/latex/contrib/titlesec/titlesec.pdf

% Below "\section" can be replaced with "\subsection" and "\subsubsection"
% in order to customize the corresponding headings



\pagestyle{fancyplain}
\fancyhead[L]{HW1: 01/31/2014}
\fancyhead[C]{University of Wisconsin-Madison}
\fancyhead[R]{Professor A. Miller}
\fancyfoot[L]{Taylor Lee}
\fancyfoot[C]{MATH 542}
\fancyfoot[R]{tdlee2@wisc.edu}
\newtheoremstyle{moo}
{10pt}
{10pt}
{}
{}
{\bfseries}
{:}
{\newline }
{}


\theoremstyle{moo}
\newtheorem{prob}{Problem}
\newtheorem*{sol}{Solution}


\begin{document}
\fontencoding {T1}
\fontfamily {put}
\fontseries {\seriesdefault}
\fontshape {\shapedefault}
%\fontsize {size} {baselineskip} 
\selectfont

\title{\usefont{T1}{put}{b}{n} Math 542-Modern Algebra II}
\date{January 31, 2014}         % used by \maketitle
\author{Taylor Lee}      % used by \maketitle                    % used by \maketitle
\maketitle                      % automatic title!

\begin{prob} (Fri Jan 24) (a) Find an integer $x$ such that
$x= 6 \mod 10$ and $x=15 \mod 21$ and $0\leq x\leq 210$.
(b) Find the smallest positive integer $y$ such that 
$y= 6 \mod 10$ and $y=15 \mod 21$ and $y= 8 \mod 11$.
\end{prob}

\begin{sol}
(a) We will use the proof of the Chinese Remainder Theorem to create a number that is the sum of two integers, $x_1, x_2$, which satisfy: $x_1 \equiv 6 \pmod{10}$ and $x_2 \equiv 15 \pmod{21}$, as well as $x_1 \equiv 0 \pmod{10}$ and $x_2 \equiv 0 \pmod{21}$. This number is hence:
\[
x = x_1 + x_2 = 6 \cdot 21 \cdot 1 + 15 \cdot 10 \cdot 19 = 2976 \equiv 36 \pmod{210}
\]
This number, $x = 36$ satisfies our system of congruences.

(b) We will use the proof of the Chinese Remainder Theorem to create a number that is the sum of three integers, $y_1,y_2,y_3$ which satisfy: $y_1 \equiv 6 \pmod{10}$, $y_2 \equiv 15 \pmod{21}$ and $y_3 \equiv 8 \pmod{11}$, as well as $y_1 \equiv 0 \pmod{21}$ and $y_1 \equiv 0 \pmod{11}$, $y_2 \equiv 0 \pmod{10}$ and $y_2 \equiv0 \pmod{21}$ and $y_3 \equiv 0 \pmod{10}$ and $y_3 \equiv 0 \pmod{11}$ repectivley. This number is hence:
\[
y = y_1 + y_2 + y_3 = 6 \cdot 231 + 15 \cdot 110 \cdot 17 + 8 \cdot 210 = 31116 \equiv 1086 \pmod{2310}
\]

This number, $y = 1086$, satisfies our system of congruences.
\end{sol}

\newpage

\begin{prob}(Fri Jan 24)
(a) Find integers $i,j$ such that there is no integer $x$ with
$x= i \mod 6$ and $x= j \mod 15$.   (b) Find all pairs $i,j$
with $i=0,1,\dots 5$ and $j=0,1,\ldots, 14$ such that there
is an integer $x$ with $x= i \mod 6$ and $x= j \mod 15$.
\end{prob}

\begin{sol}

First we will display the pairs $i,j$ for which there is no $x$ that satisfy the above system of congruences. 

\begin{tabular}{lll}

\begin{tabular}{|c|c|}

\cline{1-2}
$i$ & $j$ \\
\cline{1-2}
\cline{1-2}
0 & 1 \\
\cline{1-2}
0 & 2 \\
\cline{1-2}
0 & 4 \\
\cline{1-2}
0 & 5 \\
\cline{1-2}
0 & 7 \\
\cline{1-2}
0 & 8 \\
\cline{1-2}
0 & 10 \\
\cline{1-2}
0 & 11 \\
\cline{1-2}
0 & 13 \\
\cline{1-2}
0 & 14 \\
\cline{1-2}
1 & 0 \\
\cline{1-2}
1 & 2 \\
\cline{1-2}
1 & 3 \\
\cline{1-2}
1 & 5 \\
\cline{1-2}
1 & 6 \\
\cline{1-2}
1 & 8 \\
\cline{1-2}
1 & 9 \\
\cline{1-2}
1 & 11 \\
\cline{1-2}
1 & 12 \\
\cline{1-2}
1 & 14 \\
\cline{1-2}
\end{tabular}
&
\begin{tabular}{|c|c|}

\cline{1-2}
$i$ & $j$ \\
\cline{1-2}
\cline{1-2}
2 & 0 \\
\cline{1-2}
2 & 1 \\
\cline{1-2}
2 & 3 \\
\cline{1-2}
2 & 4 \\
\cline{1-2}
2 & 6 \\
\cline{1-2}
2 & 7 \\
\cline{1-2}
2 & 9 \\
\cline{1-2}
2 & 10 \\
\cline{1-2}
2 & 12 \\
\cline{1-2}
2 & 13 \\
\cline{1-2}
3 & 1 \\
\cline{1-2}
3 & 2 \\
\cline{1-2}
3 & 4 \\
\cline{1-2}
3 & 5 \\
\cline{1-2}
3 & 7 \\
\cline{1-2}
3 & 8 \\
\cline{1-2}
3 & 10 \\
\cline{1-2}
3 & 11 \\
\cline{1-2}
3 & 13 \\
\cline{1-2}
3 & 14 \\
\cline{1-2}
\end{tabular}

&

\begin{tabular}{|c|c|}

\cline{1-2}
$i$ & $j$ \\
\cline{1-2}
\cline{1-2}
4 & 0 \\
\cline{1-2}
4 & 2 \\
\cline{1-2}
4 & 3 \\
\cline{1-2}
4 & 5 \\
\cline{1-2}
4 & 6 \\
\cline{1-2}
4 & 8 \\
\cline{1-2}
4 & 9 \\
\cline{1-2}
4 & 11 \\
\cline{1-2}
4 & 12 \\
\cline{1-2}
4 & 14 \\
\cline{1-2}
5 & 0 \\
\cline{1-2}
5 & 1 \\
\cline{1-2}
5 & 3 \\
\cline{1-2}
5 & 4 \\
\cline{1-2}
5 & 6 \\
\cline{1-2}
5 & 7 \\
\cline{1-2}
5 & 9 \\
\cline{1-2}
5 & 10 \\
\cline{1-2}
5 & 12 \\
\cline{1-2}
5 & 13 \\
\cline{1-2}
\end{tabular}

\end{tabular}

\newpage

(b) Here we will display the pairs $i,j$ which have a solution $x$, as well as the solution $x$ itself:

\hspace{1cm}

\begin{tabular}{l  r}

\begin{tabular}{|c|c|c|}

\cline{1-3}
$i$ & $j$ & $x$ \\
\cline{1-3}
\cline{1-3}
0 & 0 & 0 \\
\cline{1-3}
0 & 3 & 18 \\
\cline{1-3}
0 & 6 & 6 \\
\cline{1-3}
0 & 9 & 24 \\
\cline{1-3}
0 & 12 & 12 \\
\cline{1-3}
1 & 1 & 1 \\
\cline{1-3}
1 & 4 & 19 \\
\cline{1-3}
1 & 7 & 7 \\
\cline{1-3}
1 & 10 & 25 \\
\cline{1-3}
1 & 13 & 13 \\
\cline{1-3}
2 & 2 & 2 \\
\cline{1-3}
2 & 5 & 20 \\
\cline{1-3}
2 & 8 & 8 \\
\cline{1-3}
2 & 11 & 26 \\
\cline{1-3}
2 & 14 & 14 \\
\cline{1-3}

\end{tabular}
&
\begin{tabular}{|c|c|c|}

\cline{1-3}
$i$ & $j$ & $x$ \\
\cline{1-3}
\cline{1-3}
3 & 0 & 15 \\
\cline{1-3}
3 & 3 & 3 \\
\cline{1-3}
3 & 6 & 21 \\
\cline{1-3}
3 & 9 & 9 \\
\cline{1-3}
3 & 12 & 27 \\
\cline{1-3}
4 & 1 & 16 \\
\cline{1-3}
4 & 4 & 4 \\
\cline{1-3}
4 & 7 & 22 \\
\cline{1-3}
4 & 10 & 10 \\
\cline{1-3}
4 & 13 & 28 \\
\cline{1-3}
5 & 2 & 17 \\
\cline{1-3}
5 & 5 & 5 \\
\cline{1-3}
5 & 8 & 23 \\
\cline{1-3}
5 & 11 & 11 \\
\cline{1-3}
5 & 14 & 29 \\
\cline{1-3}

\end{tabular}


\end{tabular}

\end{sol}



\end{document}   
