\documentclass[12pt]{article}
\usepackage{amsmath,amsfonts,amsthm}
\usepackage[fleqn]{mathtools}
\usepackage{a4wide}
\usepackage{fancyhdr} % Custom headers and footers
\usepackage{graphicx}
\usepackage{listings}
\usepackage{parskip}

\usepackage[adobe-utopia]{mathdesign}
\usepackage[T1]{fontenc}

\usepackage{titlesec} % www.ctan.org/tex-archive/macros/latex/contrib/titlesec/titlesec.pdf

% Below "\section" can be replaced with "\subsection" and "\subsubsection"
% in order to customize the corresponding headings



\pagestyle{fancyplain}
\fancyhead[L]{HW22: 03/10/2014}
\fancyhead[C]{University of Wisconsin-Madison}
\fancyhead[R]{Professor Arnold Miller}
\fancyfoot[L]{Taylor Lee}
\fancyfoot[C]{MATH 542}
\fancyfoot[R]{tdlee2@wisc.edu}
\newtheoremstyle{moo}
{10pt}
{10pt}
{}
{}
{\bfseries}
{:}
{\newline }
{}


\theoremstyle{moo}
\newtheorem*{prob}{Problem}
\newtheorem*{sol}{Solution}
\newtheorem*{theorem}{Theorem}
\def\comb(#1,#2){
\left(
\begin{array}{c}
#1 \\
#2 
\end{array}
\right)
}

\def\zz{{\mathbb Z}}
\def\su{\subseteq}


\begin{document}
\fontencoding {T1}
\fontfamily {put}
\fontseries {\seriesdefault}
\fontshape {\shapedefault}
%\fontsize {size} {baselineskip} 
\selectfont

\title{\usefont{T1}{put}{b}{n} Math 542-Modern Algebra II}
\date{March 10, 2014}         % used by \maketitle
\author{Taylor Lee}      % used by \maketitle                    % used by \maketitle
\maketitle                      % automatic title!




\begin{prob}
(Mon Mar 3) Prove for any prime $p$ and positive integer $n$
that $p$ divides $\comb(p^n,k)$ for any $k$ with $0<k<p^n$.
\end{prob}



\begin{sol}

First note that $p^n!$ can be divided by $p^c$ where 
\[
c = \frac{n \left( n+1 \right)}{2}
\]. 

Also note that $p^n!$ is the smallest number with this property. Suppose $p$ does not divide $p^n$ choose $k$, then this implies that $p^n | k!(p^n-k)!$. However, since $\comb(p^n,k)$ must be an integer, this implies that $k!(p^n-k)!$ must equal $p^n!$. This is not true, however, since $k!$ is the product of the first $k$ integers, and \[\frac{p^n!}{(p^n-k)!}\] is the product of the integers from $p^n-k + 1$ to $p^n$, which makes it clear that 

\[
k! \neq \frac{p^n!}{(p^n-k)!}
\]

And hence we have a contradiction, and $p$ must divide $\comb(p^n,k)$.

\end{sol}





\end{document}   
