\documentclass[12pt]{article}
\usepackage{amsmath,amsfonts,amsthm}
\usepackage[fleqn]{mathtools}
\usepackage{a4wide}
\usepackage{fancyhdr} % Custom headers and footers
\usepackage{graphicx}
\usepackage{listings}
\usepackage{parskip}

\usepackage[adobe-utopia]{mathdesign}
\usepackage[T1]{fontenc}

\usepackage{titlesec} % www.ctan.org/tex-archive/macros/latex/contrib/titlesec/titlesec.pdf

% Below "\section" can be replaced with "\subsection" and "\subsubsection"
% in order to customize the corresponding headings



\pagestyle{fancyplain}
\fancyhead[L]{HW14/15: 02/24/2014}
\fancyhead[C]{University of Wisconsin-Madison}
\fancyhead[R]{Professor Arnold Miller}
\fancyfoot[L]{Taylor Lee}
\fancyfoot[C]{MATH 542}
\fancyfoot[R]{tdlee2@wisc.edu}
\newtheoremstyle{moo}
{10pt}
{10pt}
{}
{}
{\bfseries}
{:}
{\newline }
{}


\theoremstyle{moo}
\newtheorem*{prob}{Problem}
\newtheorem*{sol}{Solution}
\newtheorem*{theorem}{Theorem}

\def\zz{{\mathbb Z}}
\def\su{\subseteq}


\begin{document}
\fontencoding {T1}
\fontfamily {put}
\fontseries {\seriesdefault}
\fontshape {\shapedefault}
%\fontsize {size} {baselineskip} 
\selectfont

\title{\usefont{T1}{put}{b}{n} Math 542-Modern Algebra II}
\date{Febuary 24, 2014}         % used by \maketitle
\author{Taylor Lee}      % used by \maketitle                    % used by \maketitle
\maketitle                      % automatic title!



\begin{prob}
(Mon Feb 17) Suppose for every $x\in G$ that $x^2=e$.  Prove
that $G$ is abelian.
\end{prob}

\begin{prob}
(Mon Feb 17) Suppose $H\su G$ is subgroup of index 2, i.e., 
$[G:H]=2$.  Prove that it is a normal subgroup of $G$.
\end{prob}



\begin{sol}
(a) First not that for any $x \in G$, we have 
\[
x^2 = e \Leftrightarrow x = x^{-1}
\]

So any element is equal to it's own inverse. Next, let $a,b \in G$. This implies that $ab \in G$ and

\[
\left( ab \right)^2 = e \Leftrightarrow abab = e \Leftrightarrow ab = b^{-1}a^{-1}  \Leftrightarrow ab = ba.
\]

Hence, $G$ is abelian, as $a,b$ are arbitrary and all the elements of $G$ commute.

(b) Since the index of $H$ in $G$ is 2, we can create a homomorphism $G \rightarrow G/H$ such that an element of $h \in H \subset$ maps to $0$ and an element $g \notin H$ but $g \in G$ maps to $1$. Hence, $G/H \cong \zz/ 2 \zz$, and since such a homomorphism exists, we can see that $H$ must be normal.


\end{sol}





\end{document}   
