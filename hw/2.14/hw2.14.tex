\documentclass[12pt]{article}
\usepackage{amsmath,amsfonts,amsthm}
\usepackage[fleqn]{mathtools}
\usepackage{a4wide}
\usepackage{fancyhdr} % Custom headers and footers
\usepackage{graphicx}
\usepackage{listings}
\usepackage{parskip}

\usepackage[adobe-utopia]{mathdesign}
\usepackage[T1]{fontenc}

\usepackage{titlesec} % www.ctan.org/tex-archive/macros/latex/contrib/titlesec/titlesec.pdf

% Below "\section" can be replaced with "\subsection" and "\subsubsection"
% in order to customize the corresponding headings



\pagestyle{fancyplain}
\fancyhead[L]{HW7: 02/14/2014}
\fancyhead[C]{University of Wisconsin-Madison}
\fancyhead[R]{Professor A. Miller}
\fancyfoot[L]{Taylor Lee}
\fancyfoot[C]{MATH 542}
\fancyfoot[R]{tdlee2@wisc.edu}
\newtheoremstyle{moo}
{10pt}
{10pt}
{}
{}
{\bfseries}
{:}
{\newline }
{}


\theoremstyle{moo}
\newtheorem*{prob}{Problem}
\newtheorem*{sol}{Solution}
\newtheorem*{theorem}{Theorem}

\def\zz{{\mathbb Z}}



\begin{document}
\fontencoding {T1}
\fontfamily {put}
\fontseries {\seriesdefault}
\fontshape {\shapedefault}
%\fontsize {size} {baselineskip} 
\selectfont

\title{\usefont{T1}{put}{b}{n} Math 542-Modern Algebra II}
\date{Febuary 14, 2014}         % used by \maketitle
\author{Taylor Lee}      % used by \maketitle                    % used by \maketitle
\maketitle                      % automatic title!

\begin{disc}
This is due in lecture on valentines day.  It will be graded
in class so do not hand-in.
\begin{enum}
\item Suppose $G$ is a finite abelian group which contains an
element which has non-square-free order. Prove that for some prime
$p$ it has an element of order $p^2$.
\item Suppose $a$ is an element of a finite abelian group $G$ with
order $p^2$ let $b=a^p$, let $H=<b>$ be the subgroup generated
by $b$ and suppose $K$ is a subgroup of $G$ with $K\cap H=\{e\}$.
Prove that $a$ is not an element of $HK$.
\item Suppose $G_1,G_2$ are finite abelian groups with $|G_1|$ and
$|G_2|$ relatively prime.  Show that for any subgroup $H\su G_1\times G_2$
there are subgroups $H_1\su G_1$ and $H_2\su G_2$ such that
$H=H_1\times H_2$.   (Warning: the relatively prime hypothesis is
necessary.)
\item Suppose $G_1,G_2$ are finite abelian groups with $|G_1|$ and
$|G_2|$ relatively prime.  Show that if $G_1$ and $G_2$ both have
the CP then $G_1\times G_2$ has CP.\footnote{CP is defined after Problem 4.}
\item Prove that $C_p\times C_p\times \cdots \times C_p$ has the CP.
\item Prove Problem 4.
\end{enum}
\end{disc}

\begin{sol}
If $G$ contains an element $h$ that has a non-squarefree order, then it then it has a cyclic subgroup of this order, generated by $h$.
\end{sol}



\begin{prob}
(Wed Feb 5) Prove that $Stab(ga)=g\,Stab(a)\,g^{-1}$.
\end{prob}



\begin{sol}
If we view our common group $G_1 \times G_2 \cong H_1 \times H_2$ as the product of cyclic groups of prime power, then by taking the quotient of this group by $G_1\cong H_1$, we are essentially striping away the factors of our common group which make up the cyclic prime decomposistion of $G_1\cong H_1$. It is clear that in both cases, $H$ and $G$, our remaining group will have the same structure as in the other case. In other words, if $G_1 \times G_2 \cong H_1 \times H_2$ were both isomorphic to:

\[
( \zz/p_1^{e_1} \zz ) \times \dots \times ( \zz / p_i^{e_i} \zz ) \times (\zz/ q_1^{f_1} \zz) \times \dots \times (\zz / q_j^{f_j} \zz)
\]

Where the factors of prime power $p^k$ would be isomorphic to $G_1, H_1$, then the quotient of this group by either $G_1$ or  $H_1$ would leave:

\[
(\zz/q_1^{f_1} \zz) \times \dots \times (\zz / q_j^{f_j} \zz).
\]

Which is clearly ismorphic to both $G_2$ and $H_2$.

As a counterexample, consider $G = G_1 \times G_2 = \zz \times {e}$ and $H = H_1 \times H_2 = \zz \times \zz/2\zz$. We can see that these two groups are isomorphic, if we map $H_1 \times H_2 $ by $h \mapsto 2h_1 + h_2$. In other words, double the value of $h_1$ in order to map it to the corresponding even integer, and if $h_2$ is equal to one, add $1$ to this even integer in order to get the coresponding odd integer. It is clear that $G_2 = {e} \ncong (\zz/2\zz) = H_2$, hence we have our counterexample. 
\end{sol}







\end{document}   
