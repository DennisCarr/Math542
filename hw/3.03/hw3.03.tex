\documentclass[12pt]{article}
\usepackage{amsmath,amsfonts,amsthm}
\usepackage[fleqn]{mathtools}
\usepackage{a4wide}
\usepackage{fancyhdr} % Custom headers and footers
\usepackage{graphicx}
\usepackage{listings}
\usepackage{parskip}

\usepackage[adobe-utopia]{mathdesign}
\usepackage[T1]{fontenc}

\usepackage{titlesec} % www.ctan.org/tex-archive/macros/latex/contrib/titlesec/titlesec.pdf

% Below "\section" can be replaced with "\subsection" and "\subsubsection"
% in order to customize the corresponding headings



\pagestyle{fancyplain}
\fancyhead[L]{HW17/18: 03/03/2014}
\fancyhead[C]{University of Wisconsin-Madison}
\fancyhead[R]{Professor Arnold Miller}
\fancyfoot[L]{Taylor Lee}
\fancyfoot[C]{MATH 542}
\fancyfoot[R]{tdlee2@wisc.edu}
\newtheoremstyle{moo}
{10pt}
{10pt}
{}
{}
{\bfseries}
{:}
{\newline }
{}


\theoremstyle{moo}
\newtheorem*{prob}{Problem}
\newtheorem*{sol}{Solution}
\newtheorem*{theorem}{Theorem}

\def\zz{{\mathbb Z}}
\def\su{\subseteq}


\begin{document}
\fontencoding {T1}
\fontfamily {put}
\fontseries {\seriesdefault}
\fontshape {\shapedefault}
%\fontsize {size} {baselineskip} 
\selectfont

\title{\usefont{T1}{put}{b}{n} Math 542-Modern Algebra II}
\date{March 3, 2014}         % used by \maketitle
\author{Taylor Lee}      % used by \maketitle                    % used by \maketitle
\maketitle                      % automatic title!



\begin{prob}
(Mon Feb 24) Let $R$ be a commutative ring with 1.  Let $I$ be
a maximal ideal in $R$.  Suppose $ab=0$. Prove that $a\in I$ or 
$b\in I$.
\end{prob}

\begin{prob}
(Mon Feb 24) Consider $p(x)=x^3+x+1$ as a polynomial in $\zz_2[x]$.
Suppose $p$ has a root $\alpha$ is in some field extension.  Construct
the multiplication table for 
$\zz_2[\alpha]=^{def}\{a+b\alpha+c\alpha^2 | a,b,c\in \zz_2\}$
\end{prob}

\begin{sol}
(a) Let $I$ be a maximal ideal in $R$. Since $I$ is maximal, $R/I$ is a field, and hence $R/I$ is an integral domain, and thus $I$ is a prime ideal, and hence by definition $ab \in I \implies a \in I$ or $b \in I$.


(b)

\begin{tabular}{|c|c|c|c|c|c|c|}

\hline
$0$ & $0$ & $1$ & $\alpha$ & $\alpha^2$ & $1 + \alpha$ & $1 + \aplha^2 $ & $\alpha + \alpha^2$ & $1 + \alpha + \alpha^2$ \\
\hline
$0$ & $0$ & $0$ & $0$ & $0$ & $0$ & $0$ & $0$ & $0$\\
\hline
$1$ & $0$ & $1$ & $\alpha$ & $\alpha^2$ & $1 + \alpha$ & $1 + \aplha^2 $ & $\alpha + \alpha^2$ & $1 + \alpha + \alpha^2$ \\
\hline
$\alpha$ & $0$ & $\alpha$ & $\alpha^2$ & 1 & $\alpha + \alpha^2$
\end{tabular}


(b) The multiplicative group of $\zz_{17}$ is the cyclic group $C_{16}$. This group has 8 generators.

(c) The multiplicative group of $\zz_{31}$ is the cyclic group $C_{30}$. This group has 8 generators.

\end{sol}





\end{document}   
