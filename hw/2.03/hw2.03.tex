\documentclass[12pt]{article}
\usepackage{amsmath,amsfonts,amsthm}
\usepackage[fleqn]{mathtools}
\usepackage{a4wide}
\usepackage{fancyhdr} % Custom headers and footers
\usepackage{graphicx}
\usepackage{listings}
\usepackage{parskip}

\usepackage[adobe-utopia]{mathdesign}
\usepackage[T1]{fontenc}

\usepackage{titlesec} % www.ctan.org/tex-archive/macros/latex/contrib/titlesec/titlesec.pdf

% Below "\section" can be replaced with "\subsection" and "\subsubsection"
% in order to customize the corresponding headings



\pagestyle{fancyplain}
\fancyhead[L]{HW2: 02/03/2014}
\fancyhead[C]{University of Wisconsin-Madison}
\fancyhead[R]{Professor A. Miller}
\fancyfoot[L]{Taylor Lee}
\fancyfoot[C]{MATH 542}
\fancyfoot[R]{tdlee2@wisc.edu}
\newtheoremstyle{moo}
{10pt}
{10pt}
{}
{}
{\bfseries}
{:}
{\newline }
{}


\theoremstyle{moo}
\newtheorem*{prob}{Problem}
\newtheorem*{sol}{Solution}
\newtheorem*{theorem}{Theorem}

\def\zz{{\mathbb Z}}



\begin{document}
\fontencoding {T1}
\fontfamily {put}
\fontseries {\seriesdefault}
\fontshape {\shapedefault}
%\fontsize {size} {baselineskip} 
\selectfont

\title{\usefont{T1}{put}{b}{n} Math 542-Modern Algebra II}
\date{Febuary 3, 2014}         % used by \maketitle
\author{Taylor Lee}      % used by \maketitle                    % used by \maketitle
\maketitle                      % automatic title!

\begin{prob}
(Mon Jan 27) Prove that for any $n$ there is only one Abelian group (up to 
isomorphism) of size $n$ iff $n$ is square-free.  Square-free
mean that no $p^2$ divides $n$ for $p$ a prime.
\end{prob}

We will make use of the following theorem, which was proved in class.

\begin{theorem}
$\zz_n\times\zz_m \cong \zz_{nm}$ iff $n,m$ are relatively prime.
\end{theorem}

\begin{sol}
Suppose that $G$ is an Abelian of order $n$ where $n$ is square-free. By the fundemental theorem of finite Abelian groups, we can express $G$ as the direct sum of finite cyclic subgroups of prime power order. In this special case:

\[
G \cong \zz_{p_1} \times \zz_{p_2} \times \dots \times \zz_{p_k}
\]

Each $p_n$ is unique. Naturally, $n = p_1 \times \dots \times p_k$ Hence, we can see that with a simple induction,

\[
\zz_{p_1} \times \zz_{p_2} \times \dots \times \zz_{p_n} \cong \zz_{p_1p_2} \times \dots \times \zz_{p_k} \cong \zz_{n}
\]

Where, again, $n$ is the product of $p_1 \dots p_k$.

Conversely, suppose the order $n$ of an Abelian $G$ is not squarefree. Then we can factor $G$ as:

\[
G \cong G_{p_1} \times \dots \times G_{p_k}
\]

Where each $G_k$ is the $p_k$-subgroup of $G$. Let $p_j$ be our prime whose square (at least) divides $n$. Then within $G_{p_j}$, we can a subgroup that is isomorphic to either:

\[
\left( G_{p_j} \geq \zz_{p_j} \times \zz_{p_j} \right) \text{or} \, \left( G_{p_j} \geq \zz_{p^2_j} \right).
\]

And by the theorem above, these two subgroups are distinct, and hence we can have more than one Abelian group of order $n$.

\end{sol}





\end{document}   
