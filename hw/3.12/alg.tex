% LaTex2e
\documentclass[12pt]{article}
\usepackage{amssymb,latexsym,hyperref,verbatim}

\def\nn{{\mathbb N}}
\def\qq{{\mathbb Q}}
\def\div{\slash}
\def\span{{\rm span}}
\def\al{\alpha}
\def\be{\beta}
\def\la{\lambda}
\def\si{\sigma}
\def\zz{{\mathbb Z}}
\def\rr{{\mathbb R}}
\def\cc{{\mathbb C}}
\def\ss{{\cal G}}
\def\ii{{\cal F}}
\def\al{\alpha}
\def\be{\beta}
\def\si{\sigma}
\def\ga{\gamma}
\def\de{\delta}
\def\pr{^\prime}
\def\span{{ \rm span }}
\def\isom{\simeq}
\def\su{\subseteq}
\def\st{\;:\;}
\def\aut{{\rm aut}}
\def\fix{{\rm fix}}
\def\normal{\triangleleft}
\def\pf{{\bf Proof:}\par\noindent}
\def\qed{\par\noindent {\bf :foorP}}
\def\sm{\setminus}
\def\mod{\mbox{ mod }}

\def\ra{\rangle}
\def\la{\langle}

\def\comb(#1,#2){
\left(
\begin{array}{c}
#1 \\
#2 
\end{array}
\right)
}



%beginthms

\newtheorem{theorem}{Theorem}
\newtheorem{theor}[theorem]{Theorem}
\newtheorem{thm}[theorem]{Theorem}
\newtheorem{lemma}[theorem]{Lemma}
\newtheorem{define}[theorem]{Definition}
\newtheorem{remark}[theorem]{Remark}
\newtheorem{prop}[theorem]{Proposition}
\newtheorem{ques}[theorem]{Question}
\newtheorem{quest}[theorem]{Question}
\newtheorem{cor}[theorem]{Corollary}
\newtheorem{examp}[theorem]{Example}

%endthms

\newtheorem{prob}{Problem}



\newenvironment{enum}
{\begin{list}{(\alph{enumi})}
{\usecounter{enumi}\setlength{\rightmargin}{\leftmargin}}}
{\end{list}}


\begin{document}

\noindent  Last revised: \today

\begin{flushright}
  A.Miller $\;\;\;\;$  M542 \\
  www.math.wisc.edu/$\sim$miller/\\
\end{flushright}

%thmcutbegin

\bigskip
Each Problem is due one week from the date it is assigned.
Do not hand them in early.
Please put them on the desk in front of the room at the
beginning or end of class. Include the statement of the problem
as part of your solution.

The date the problem was assigned in class is in parentheses.

\bigskip

%thmcutend

\begin{theorem}
(Chinese remainder theorem) Given $n,m$ relatively prime integers for
every $i,j\in\zz$ there is an $x\in\zz$ such that
$x=i$ mod  $n$ and $x=j$ mod  $m$.
\end{theorem}



\begin{prob} (Fri Jan 24) (a) Find an integer $x$ such that
$x= 6 \mod 10$ and $x=15 \mod 21$ and $0\leq x\leq 210$.
(b) Find the smallest positive integer $y$ such that 
$y= 6 \mod 10$ and $y=15 \mod 21$ and $y= 8 \mod 11$.
\end{prob}

\begin{prob}(Fri Jan 24)
(a) Find integers $i,j$ such that there is no integer $x$ with
$x= i \mod 6$ and $x= j \mod 15$.   (b) Find all pairs $i,j$
with $i=0,1,\dots 5$ and $j=0,1,\ldots, 14$ such that there
is an integer $x$ with $x= i \mod 6$ and $x= j \mod 15$.
\end{prob}


\begin{theorem}
$\zz_n\times\zz_m\isom \zz_{nm}$ iff $n,m$ are relatively prime.
\end{theorem}

\begin{lemma}
Suppose $n,m$ are relatively prime, $G$ is a finite abelian group such that
$x^{nm}=e$ for every $x\in G$.  Let $G_n=\{x\in G\st x^n=e\}$ and
$G_m=\{x\in G\st x^m=e\}$.  Then
\begin{itemize}
\item $G_n$ and $G_m$ are subgroups of $G$,
\item $G_n\cap G_m=\{e\}$,
\item $G_nG_m=G$, and therefore
\item $G\isom G_n\times G_m$
\end{itemize}
\end{lemma}

\begin{cor} (Decomposition into $p$-groups)
Suppose $G$ is an abelian group and $|G|=p_1^{i_1}\cdot p_2^{i_2}\cdots p_n^{i_n}$
where $p_1<p_2<\cdots <p_n$ are primes.  Then
$$G\isom G_1\times G_2\times\cdots\times G_n$$ 
where for each $j$ if $x\in G_j$ then $x^{n_j}=e$ where $n_j=p_j^{i_j}$.
\end{cor}

\begin{prob}
(Mon Jan 27) Prove that for any $n$ there is only one abelian group (up to 
isomorphism) of size $n$ iff $n$ is square-free.  Square-free
mean that no $p^2$ divides $n$ for $p$ a prime.
\end{prob}

\begin{lemma}
Suppose $G$ is a finite abelian $p$-group and $a\in G$ has maximum
order, then there exists a subgroup $K\su G$ such that
\begin{itemize}
\item $\la a\ra\cdot K=G$ and 
\item $\la a\ra\cap K=\{e\}$.
\end{itemize}
\end{lemma}

The proof given in class is like the one in Gallian or Judson.

\begin{theorem}
Any finite abelian group is isomorphic to a product of cyclic
groups each of which has prime-power order.
\end{theorem}

\begin{prob}(Wed Jan 29)
Let $G$ be a finite abelian group.  Prove that the following are equivalent
\begin{enumerate}
\item For every subgroup $H$ of $G$ there is a subgroup $K$ of $G$ with
$HK=G$ and $H\cap K=\{e\}$. 
\item Every  element of $G$ has square-free order.
\end{enumerate}
\end{prob}

%thmcutbegin

Hint: Polya's Dictum: ``If you can't do a problem, then there is an
easier problem you can't do.  Find it.''

Lets call property (1) the Complementation Property for $G$
or CP for short.  Here are some easier problems:
\begin{enum}
\item Prove that $C_{p^2}$ fails to have CP.
\item Prove that $C_p\times C_p$ has CP.
\item Let $|G|$ and $|H|$ be
relatively prime.  Prove that $G\times H$ has CP iff both
$G$ and $H$ have CP.
\end{enum}

%thmcutend

\begin{theorem}(Uniqueness)
Suppose 
$$C_{p^{n_1}}\times C_{p^{n_1}}\times\cdot\times C_{p^{n_k}}\isom
C_{p^{m_1}}\times C_{p^{m_1}}\times\cdot\times C_{p^{m_l}}$$
where $n_1\geq n_2\geq \cdots n_k\geq 1$ and 
$m_1\geq m_2\geq \cdots m_l\geq 1$.  Then $k=l$ and $n_i=m_i$ for
all $i$.
\end{theorem}

\begin{prob}(Fri Jan 31)
How many abelian groups of order 144 are there up to isomorphism?
Explain.
\end{prob}

\begin{prob}
(Mon Feb 3) Suppose $G_1,G_2,H_1,H_2$ are finite abelian groups,
$G_1\times G_2\isom H_1\times H_2$ and $G_1\isom H_1$.
Prove that $G_2\isom H_2$.
\par Give a counterexample if the word finite is dropped, i.e.,
$G_1\times G_2\isom H_1\times H_2$ and $G_1\isom H_1$ but $G_2$
is not isomorphic to $H_2$.   
\end{prob}

%thmcutbegin

\noindent 
I do not know if problem 6 is true or false for finite non-abelian groups.

%thmcutend

\bigskip
For the group $G$ acting on the set $X$
the orbit of $a\in X$ is
$$orb(a)=^{def}\{ga\st g\in G\}\su X.$$

\begin{prop} Orbits are either disjoint or the same.
\end{prop}

\begin{prob}
(Wed Feb 5) Prove or disprove:
\par For any finite abelian groups $G_1$ and $G_2$ with
subgroups, $H_1\su G_1$ and $H_2\su G_2$ such that $H_1\isom H_2$,
if $G_1/H_1\not\isom G_2/H_2$ then $G_1\not\isom G_2$.
\end{prob}

For a given group action of group $G$ on set
$X$, define $Stab(a)=\{g\in G\st ga=a\}$ for each $a\in X$.
Called stabilizer or fixed subgroup.

\begin{prop}
$Stab(a)$ is a subgroup of $G$.
\end{prop}

\begin{prob}
(Wed Feb 5) Prove that $Stab(ga)=g\,Stab(a)\,g^{-1}$.
\end{prob}

For $H\su G$ a subgroup the index of $H$, $[G:H]$ is the
number of $H$-cosets, $|\{gH\st g\in G\}|$.  Lagrange's
Theorem says $|G|=[G:H]\cdot |H|$.

\begin{prop}
(Orbit-stabilizer formula) $|orb(a)|=[G:Stab(a)]$.
\end{prop}

The conjugacy action of $G$ on $G$ is given by $(g,h) \to ghg^{-1}$.
Under this action the orbits are called the conjugacy classes.
$Z(G)$ the center of $G$ is the subgroup of all elements of
$G$ which commute with every other element of $g$.  Equivalently
it is the set of elements of $G$ with orbits (conjugacy classes) 
of size one. $C(g)=Stab(g)$ is called the centralizer subgroup of $g$.

\begin{theorem}
(Class formula) If $conj(g_1),\;\cdots, \; conj(g_n)$ are the
conjugacy classes of size greater
than one, then
$$|G|=|Z(G)| +\sum_{k=1}^n [G:C(g_k)]$$
\end{theorem}

\begin{theorem}
(Cauchy) If $p$ is a prime which divides $|G|$, then $G$ has
an element of order $p$.
\end{theorem}

\begin{prob}
This is due in lecture on valentines day.  It will be graded
in class so do not hand-in.

\par\noindent (a) Suppose $G$ is a finite abelian group which contains an
element which has non-square-free order. Prove that for some prime
$p$ it has an element of order $p^2$.
\par\noindent (b) Suppose $a$ is an element of a finite abelian group $G$ with
order $p^2$ let $b=a^p$, let $H=<b>$ be the subgroup generated
by $b$ and suppose $K$ is a subgroup of $G$ with $K\cap H=\{e\}$.
Prove that $a$ is not an element of $HK$.
\par\noindent (c) Suppose $G_1,G_2$ are finite abelian groups with $|G_1|$ and
$|G_2|$ relatively prime.  Show that for any subgroup $H\su G_1\times G_2$
there are subgroups $H_1\su G_1$ and $H_2\su G_2$ such that
$H=H_1\times H_2$.   (Warning: the relatively prime hypothesis is
necessary.)
\par\noindent (d) Suppose $G_1,G_2$ are finite abelian groups with $|G_1|$ and
$|G_2|$ relatively prime.  Show that if $G_1$ and $G_2$ both have
the CP then $G_1\times G_2$ has CP.\footnote{CP is defined after Problem 4.}
\par\noindent (e) Prove that $C_p\times C_p\times \cdots \times C_p$ has the CP.
\par\noindent (f) Prove Problem 4.

\end{prob}

\begin{cor}
Groups of order $p^2$ are abelian.
\end{cor}

\begin{thm}
(Sylow 1) If $G$ is a finite group and $p^n$ divides $|G|$,
then there exists a subgroup $H\su G$ with $|H|=p^n$.
\end{thm}

\begin{prop}
Any two n-cycles in $S_N$ are conjugates.  If $\tau=c_1c_2\cdots c_n$
and $\rho=c_1\pr c_2\pr \cdots c_n\pr$ are
disjoint cycle decomposition with $|c_i|=|c_i\pr|$ all $i$, then
$\tau$ and $\rho$ are conjugates.  Similarly for the converse.
\end{prop}

\begin{prob}
(Mon Feb 10) Prove for any $n\geq 3$ that
$Z(S_n)=\{id\}$.
\end{prob}

\begin{define}
$H\su G$ is a $p$-subgroup iff its order is a power of $p$. $P\su G$
is a $p$-Sylow subgroup of $G$ iff $|P|=p^n$ where $|G|=p^nm$ and
$p$ does not divide $m$.
\end{define}

\begin{lemma}
Suppose $P$ is a $p$-Sylow subgroup of $G$, $g\in G$ has
order a power of $p$, and $gPg^{-1}=P$.  Then $g\in P$.
\end{lemma}

\begin{thm}
(Sylow 2) If $G$ is a finite group, $H$ a p-subgroup of $G$, and $P$
a $p$-Sylow subgroup of $G$, then there exists $g\in G$ such that
$H\su gPg^{-1}$.
\end{thm}

\begin{cor} Let $G$ be a finite group such that $p$ divides $|G|$.
\par (a) Any $p$-subgroup of $G$ is contained in a $p$-Sylow subgroup of $G$.
\par (b) Any two $p$-Sylow subgroups of $G$ are conjugates.
\par (c) Any two $p$-Sylow subgroups of $G$ are isomorphic.
\par (d) A $p$-Sylow subgroup of $G$ is normal iff it is the only
$p$-Sylow subgroup of $G$.
\end{cor}

\begin{thm}
(Sylow 3) If $|G|=p^nm$ where $p$ does not divide $m$ and
$n(p)$ is the number of $p$-Sylow subgroups of $G$, then:
\par (a) $n(p)=[G:N(P)]$ for any $P$ a $p$-Sylow subgroup of $G$,
\par (b) $n(p)$ divides $m$, and
\par (c) $n(p)=1$ mod $p$
\end{thm}

\begin{prob}(Wed Feb 12)
\par (a) Prove that there are no simple groups of order
either $575$ or $272$.
\par (b) For any prime $p$ prove there are no simple groups
of order $p(p-1)$ or $p(p+2)$.
\end{prob}

\begin{thm}
If $p<q$ are primes and $q$ is not $1$ mod $p$, then every
group of order $pq$ is abelian.
\end{thm}

\begin{prob}
(Fri Feb 14) Question (August J.) Suppose every subgroup
of finite group $G$ is a normal subgroup.  Must $G$ be abelian?
\end{prob}

\begin{prob}
(Fri Feb 14)
\par (a) Suppose $P$ is a $p$-Sylow subgroup of $G$ and
$H$ a subgroup such that $P\normal H$ and $H\normal G$.
Prove that $P\normal G$.
\par (b) If $K\normal H$ and $H\normal G$, does it follow
that $K\normal G$?  Show that the answer is No.
Consider $G=S_4$, $K=\{id,\si\}$ where $\si=(12)(34)$ and
$H=\{id,\si,\tau,\rho\}$ where $\tau$ and $\rho$ are conjugates
of $\si$.  Determine what $\tau$ and $\rho$ are and show
that $K\normal H$ and $H\normal G$, but $K$ is not a normal subgroup
of  $G$.
\end{prob}

\begin{thm}
$aut(\zz_p,+_p)$ is isomorphic to 
$(\zz_p\sm\{0\},\times_p)$ the multiplicative group
of nonzero elements.
\end{thm}

\begin{examp}
If $p<q$ are primes and $q=1$ mod $p$, then there is a twisted product of
$\zz_p$ and $\zz_q$ which has order $pq$ and is not abelian.
\end{examp}

\begin{prob}
(Mon Feb 17) Suppose for every $x\in G$ that $x^2=e$.  Prove
that $G$ is abelian.
\end{prob}

\begin{prob}
(Mon Feb 17) Suppose $H\su G$ is subgroup of index 2, i.e., 
$[G:H]=2$.  Prove that it is a normal subgroup of $G$.
\end{prob}

\begin{thm}
Suppose that $p(x)$ is a polynomial over the field $F$ and for some $\al\in F$
$\;p(\al)=0$.  Then $p(x)=(x-\al)q(x)$ for
some polynomial $q(x)$.
\end{thm}

\begin{cor}
Any polynomial $p\in F[x]$ of degree $\leq n$ with more than $n$ roots
must be identically zero.
\end{cor}

\begin{thm}
Let the exponent of $G$ be the least $n$ such that
$x^n=e$ for every $x\in G$.  If $G$ is finite abelian group then 
$G$ is cyclic iff $exp(G)=|G|$.
\end{thm}

\begin{cor}
The multiplicative group of a finite field is cyclic.
\end{cor}

\begin{prob}(Wed Feb 19)
For $F$ a finite field call $a\in F$ a generator of $F$ iff
every nonzero element of $F$ is a power of $a$.
\par (a) Find a generator of $\zz_7$.
\par (b) How many generators does $\zz_{17}$ have?
\par (c) How many generators does $\zz_{31}$ have?
\end{prob}

\begin{thm}
If $p<q$ are primes and $q=1$ mod $p$, then up to isomorphism
there is a unique nonabelian group of order $pq$.
\end{thm}


For elementary results on vector spaces see:

\url{http://www.math.wisc.edu/~miller/old/m542-00/vector.pdf}



\begin{lemma}
(Exchange Lemma) Suppose $\span(A\cup B)=V$ and $a\notin \span(A)$.
Then there exists $b\in B$ such that $\span(A\cup\{a\}\cup(B\sm\{b\}))=V$.
\end{lemma}

\begin{thm}
Every vector space has a basis.  Any two bases have the
same cardinality.  Any set of $n+1$ vectors in a vector
space of dimension $n$ is linearly dependent.
\end{thm}

\begin{cor}
Any finite field $F$ of characteristic $p$ has cardinality
$p^n$ for some integer $n$.
\end{cor}

\begin{prob}
(Fri Feb 21) Prove that $v_1,v_2,\ldots,v_n$ are linearly dependent
iff $v_1=\vec{0}$ or $v_{i+1}\in\span\{v_1,v_2,\ldots,v_i\}$
for some $i$ with $1\leq i<n$.
\end{prob}


\begin{thm}
(Kronecker) If $p(x)\in F[x]$ is a non-constant polynomial, then
there exists a field $E\supseteq F$ and $\al\in E$ with
$p(\al)=0$.
\end{thm}

\begin{prob}
(Mon Feb 24) Let $R$ be a commutative ring with 1.  Let $I$ be
a maximal ideal in $R$.  Suppose $ab=0$. Prove that $a\in I$ or 
$b\in I$.
\end{prob}

\begin{prob}
(Mon Feb 24) Consider $p(x)=x^3+x+1$ as a polynomial in $\zz_2[x]$.
Suppose $p$ has a root $\al$ is in some field extension.  Construct
the multiplication table for 
$$\zz_2[\al]=^{def}\{a+b\al+c\al^2 \st a,b,c\in \zz_2\}$$
\end{prob}


\begin{cor}(Kronecker)
If $p(x)\in F[x]$ is a polynomial of degree $n$, then
there exists a field $E\supseteq F$ and $\al_i\in E$ such that
$$p(x)=a(x-\al_1)(x-\al_2)\cdots (x-\al_n)$$
\end{cor}

\begin{thm}
If $p(x)\in F[x]$ is irreducible and $\al,\be$ are roots in
some extension fields of $F$ then $F(\al)$ and $F(\be)$ are
isomorphic via an isomorphism which fixes $F$.
\end{thm}

\begin{cor}
If  $p(x)\in F[x]$ is irreducible and splits in an extension field $E$
of $F$ then the multiplicity of each root of $p$ is the same. 
\end{cor}


\begin{prob}
(Wed Feb 26) Let $\al$ be transcendental over $\zz_2$.  Let
$F=\zz_2(\al)$ and let $p(x)=x^2-\al$.  
\par (a) Prove that $p$ is irreducible over $F$.
\par (b) Prove that if $\be$ is a root of $p$ in some
some extension field, then $p(x)=(x-\be)^2$.
\par (c) Suppose that $F$ is a finite field of characteristic $2$.
Prove that for every $a\in F$ there is a $b\in F$ such that $b^2=a$.
\par (d)  Suppose that $F$ is a finite field of odd characteristic.
Prove that there exists $a\in F$ for every $b\in F$ such that $b^2\neq a$.
\par (e) Find a field $F$ and an irreducible polynomial $p(x)$ of
degree three such that in any extension field in which $p$ splits
there exist a $\be$ such that $p(x)=(x-\be)^3$.
\end{prob}

\begin{thm}
The formal derivative for an abstract polynomial $f(x)\in F[x]$ satisfies
the usual derivative laws:
\begin{enum}
\item If $a\in F$ and $f\in F[x]$, then $(af)\pr=af\pr$.
\item If $f,g\in F[x]$, then $(f+g)\pr=f\pr+g\pr$.
\item If $f,g\in F[x]$, then $(fg)\pr=f\pr g+fg\pr$.
\end{enum}
\end{thm}

\begin{prob}
(Fri Feb 28) Prove that the formal derivative for polynomials
in $F[x]$ satisfies 
\begin{enum}
\item The power rule: $(f^n)\pr =n(f^{n-1})f\pr$
\item The chain rule: $f(g(x))\pr=f\pr(g(x))g\pr(x)$
\end{enum}
\end{prob}

\begin{thm}
For any $\al\in F$ and $f\in F[x]$ $\;\;\;\;$
\par $\al$ is repeated root of $f$ iff it is a root of $f\pr$.
\end{thm}

\begin{cor}
The roots of an irreducible polynomial in a field of characteristic
zero, are always distinct.
\end{cor}

\begin{lemma}
If $E$ is any field of characteristic $p$, then for any $\al,\be\in E$
$$(\al+\be)^{p^n}=\al^{p^n}+\be^{p^n}$$
\end{lemma}

\begin{thm}
For any $p^n$ and there is a 
field $F$ with $|F|=p^n$.
\end{thm}

\begin{prob}
(Mon Mar 3) Prove for any prime $p$ and positive integer $n$
that $p$ divides $\comb(p^n,k)$ for any $k$ with $0<k<p^n$.
\end{prob}

\begin{define}
For fields $F\su E$ define $[E:F]$ to be the dimension
of $E$ viewed as a vector space over $F$.
\end{define}

\begin{thm}
For fields $F\su K\su E$
$$[E:F]=[E:K]\cdot [K:F]$$
\end{thm}

\begin{thm}
For $p(x)\in F[x]$ irreducible and $\al$ a root of $p$ in
some extension field, $[F[\al]:F]$ is the degree of $p$.
\end{thm}

\begin{thm}
If $E\supseteq F$ is the splitting field of some polynomial in $F[x]$,
then $[E:F]$ is finite.
\end{thm}

\begin{thm}
If $[E:F]$ is finite and $\al\in E$, then there is an irreducible
polynomial $p\in F[x]$ with $p(\al)=0$.
\end{thm}

\begin{prob}
(Wed Mar 5) $p$ is a prime and $n$ a positive integer.
Prove:
\par (a) If $F$ is a field such that $|F|=p^n$ and $m$ is a
positive integer then there is a field $E$ with $F\su E$
and $E=p^{nm}$.
\par (b) If $F\su E$ are fields, $|F|=p^n$
and $|E|=p^N$, then $n$ divides $N$.
\end{prob}

\begin{define}
$\al$ is algebraic over $F$ iff it is the root of a nontrivial polynomial
in $F[x]$.  A field $K$ is algebraically closed iff every nonconstant 
polynomial $f\in K[x]$ has a root in $K$.
\end{define}

\begin{thm}
If $F\su E$ are fields define
$$K=\{\al\in E\st \al \mbox{ is algebraic over } F\}$$
Then $K$ is a field and $F\su K\su F$.
\end{thm}

Steinitz proved that every field $F$ is a subfield of an algebraically
closed field $K$.  This requires the Axiom of Choice.

\begin{thm}
Suppose $F\su K$ and $K$ is algebraically closed.  Let
$E$ be the elements of $K$ which are algebraic over $F$.  Then
$E$ is algebraically closed.
\end{thm}



\end{document}
