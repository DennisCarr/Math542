\documentclass[12pt]{article}
\usepackage{amsmath,amsfonts,amsthm}
\usepackage[fleqn]{mathtools}
\usepackage{a4wide}
\usepackage{fancyhdr} % Custom headers and footers
\usepackage{graphicx}
\usepackage{listings}
\usepackage{parskip}

\usepackage[adobe-utopia]{mathdesign}
\usepackage[T1]{fontenc}

\usepackage{titlesec} % www.ctan.org/tex-archive/macros/latex/contrib/titlesec/titlesec.pdf

% Below "\section" can be replaced with "\subsection" and "\subsubsection"
% in order to customize the corresponding headings



\pagestyle{fancyplain}
\fancyhead[L]{HW16: 02/28/2014}
\fancyhead[C]{University of Wisconsin-Madison}
\fancyhead[R]{Professor Arnold Miller}
\fancyfoot[L]{Taylor Lee}
\fancyfoot[C]{MATH 542}
\fancyfoot[R]{tdlee2@wisc.edu}
\newtheoremstyle{moo}
{10pt}
{10pt}
{}
{}
{\bfseries}
{:}
{\newline }
{}


\theoremstyle{moo}
\newtheorem*{prob}{Problem}
\newtheorem*{sol}{Solution}
\newtheorem*{theorem}{Theorem}

\def\zz{{\mathbb Z}}
\def\su{\subseteq}


\begin{document}
\fontencoding {T1}
\fontfamily {put}
\fontseries {\seriesdefault}
\fontshape {\shapedefault}
%\fontsize {size} {baselineskip} 
\selectfont

\title{\usefont{T1}{put}{b}{n} Math 542-Modern Algebra II}
\date{Febuary 26, 2014}         % used by \maketitle
\author{Taylor Lee}      % used by \maketitle                    % used by \maketitle
\maketitle                      % automatic title!


\begin{prob}
(Fri Feb 21) Prove that $v_1,v_2,\ldots,v_n$ are linearly dependent
iff $v_1=\vec{0}$ or $v_{i+1}\in\span\{v_1,v_2,\ldots,v_i\}$
for some $i$ with $1\leq i<n$.
\end{prob}


\begin{sol}
Suppose $v_{i+1}\in\span\{v_1,v_2,\ldots,v_i\}$. Then there exist scalars $a_1, a_2, \ldots, a_i$ such that:

\[
v_{i+1} = a_1v_1 + a_2v_2 + \ldots + a_iv_i
\]

And hence:

\[
a_1v_1 + a_2v_2 + \ldots + a_iv_i - v_{i+1} = \vec{0}
\]

And thus the set $\{v_1,v_2,\ldots,v_n\}$ is not linearly independent. If there exists a $v_i = \vec{0} \in \{v_1,v_2,\ldots,v_n\}$, then for any other vector $v_k \in \{v_1,v_2,\ldots,v_n\}$, we can use scalar $a_k = 0$ to get:

\[

v_i + a_kv_k = \vec{0} + 0*v_k = \vec{0}

\]

Which shows that is this circumstance our set is linearly dependent.

Conversley, suppose that the vectors $v_1,v_2,\ldots,v_n$ are linearly dependent. Then we can see that any vector can be formed by a linear combination of our other vectors or one of these vectors must be the zero vector as follows: Let $a_1v_1 + a_2v_2 + \ldots + a_iv_i = 0$ be as from the definition. Then for any $v_k$ in this set, we have:
\[
v_k = - \frac{a_1}{a_k}v_1 - \frac{a_2}{a_k}v_2 - \ldots - \frac{a_i}{a_k}v_i
\] 

and hence $v_k \in\span\{v_1,v_2,\ldots,v_{i-1}\}$, unless our linear dependence is a result of one of our vectors being the zero vector, and all other scalars in our statement of linear independence are equal to zero, in which event the other case of our theorem would be true, and thus our result is proven.

\end{sol}





\end{document}   
