% LaTex2e
\documentclass[12pt]{article}
\usepackage{amssymb,latexsym}
\usepackage{amsmath,amsfonts,amsthm}
\usepackage[fleqn]{mathtools}
\usepackage{fourier}
\usepackage[T1]{fontenc}
\usepackage{a4wide}
\usepackage{fancyhdr} % Custom headers and footers
\usepackage{graphicx}
\usepackage{listings}
\usepackage{parskip}


\usepackage{titlesec}

%  \oddsidemargin 5pt \evensidemargin 5pt \marginparwidth 68pt
%  \marginparsep 10pt \topmargin 10pt \headheight 10pt
%  \headsep 15pt \footskip 35pt
%  \textheight = 625pt  \textwidth 440pt \columnsep 10pt \columnseprule 5pt
%\def\prob{\par\bigskip\noindent}

\def\nn{{\mathbb N}}
\def\qq{{\mathbb Q}}
\def\div{\slash}
\def\span{{\rm span}}
\def\al{\alpha}
\def\be{\beta}
\def\la{\lambda}
\def\si{\sigma}
\def\zz{{\mathbb Z}}
\def\rr{{\mathbb R}}
\def\cc{{\mathbb C}}
\def\ss{{\cal G}}
\def\ii{{\cal F}}
\def\al{\alpha}
\def\be{\beta}
\def\si{\sigma}
\def\ga{\gamma}
\def\de{\delta}
\def\pr{\prime}
\def\span{{ \rm span }}
\def\isom{\simeq}
\def\su{\subseteq}
\def\st{\;:\;}
\def\aut{{\rm aut}}
\def\fix{{\rm fix}}
\def\normal{\triangleleft}
\def\pf{{\bf Proof:}\par\noindent}
\def\qed{\par\noindent {\bf :foorP}}

%\newtheorem{theorem}{Theorem}[section]

\newtheorem{theorem}{Theorem}
\newtheorem{theor}[theorem]{Theorem}
\newtheorem{lemma}[theorem]{Lemma}
\newtheorem{define}[theorem]{Definition}
\newtheorem{remark}[theorem]{Remark}
\newtheorem{prop}[theorem]{Proposition}
\newtheorem{ques}[theorem]{Question}
\newtheorem{quest}[theorem]{Question}
\newtheorem{cor}[theorem]{Corollary}

\newtheorem{prob}{Problem}

\newenvironment{enum}
{\begin{list}{(\alph{enumi})}
{\usecounter{enumi}\setlength{\rightmargin}{\leftmargin}}}
{\end{list}}


\begin{document}

\noindent  Last revised: \today

\begin{flushright}
  A.Miller $\;\;\;\;$  M542 \\
  http://www.math.wisc.edu/$\sim$miller/\\
\end{flushright}


\bigskip
Each Problem is due one week from the date it is assigned.
Do not hand them in early.
Please put them on the desk in front of the room at the
beginning or end of class. Include the statement of the problem
as part of your solution.

The date the problem was assigned in class is in parantheses.

\bigskip

\begin{theorem}
(Chinese remainder theorem) Given $n,m$ relatively prime integers for
every $i,j\in\zz$ there is an $x\in\zz$ such that
$x=i$ mod  $n$ and $x=j$ mod  $m$.
\end{theorem}

\def\mod{\mbox{ mod }}

\begin{prob} (Fri Jan 24) (a) Find an integer $x$ such that
$x= 6 \mod 10$ and $x=15 \mod 21$ and $0\leq x\leq 210$.
(b) Find the smallest positive integer $y$ such that 
$y= 6 \mod 10$ and $y=15 \mod 21$ and $y= 8 \mod 11$.
\end{prob}

\begin{prob}(Fri Jan 24)
(a) Find integers $i,j$ such that there is no integer $x$ with
$x= i \mod 6$ and $x= j \mod 15$.   (b) Find all pairs $i,j$
with $i=0,1,\dots 5$ and $j=0,1,\ldots, 14$ such that there
is an integer $x$ with $x= i \mod 6$ and $x= j \mod 15$.
\end{prob}


\begin{theorem}
$\zz_n\times\zz_m\isom \zz_{nm}$ iff $n,m$ are relatively prime.
\end{theorem}

\begin{lemma}
Suppose $n,m$ are relatively prime, $G$ is a finite abelian group such that
$x^{nm}=e$ for every $x\in G$.  Let $G_n=\{x\in G\st x^n=e\}$ and
$G_m=\{x\in G\st x^m=e\}$.  Then
\begin{itemize}
\item $G_n$ and $G_m$ are subgroups of $G$,
\item $G_n\cap G_m=\{e\}$,
\item $G_nG_m=G$, and therefore
\item $G\isom G_n\times G_m$
\end{itemize}
\end{lemma}

\begin{cor} (Decomposition into $p$-groups)
Suppose $G$ is an abelian group and $|G|=p_1^{i_1}\cdot p_2^{i_2}\cdots p_n^{i_n}$
where $p_1<p_2<\cdots <p_n$ are primes.  Then
$$G\isom G_1\times G_2\times\cdots\times G_n$$ 
where for each $j$ if $x\in G_j$ then $x^{n_j}=e$ where $n_j=p_j^{i_j}$.
\end{cor}

\begin{prob}
(Mon Jan 27) Prove that for any $n$ there is only one abelian group (up to 
isomorphism) of size $n$ iff $n$ is square-free.  Square-free
mean that no $p^2$ divides $n$ for $p$ a prime.
\end{prob}

\def\ra{\rangle}
\def\la{\langle}

\begin{lemma}
Suppose $G$ is a finite abelian $p$-group and $a\in G$ has maximum
order, then there exists a subgroup $K\su G$ such that
\begin{itemize}
\item $\la a\ra\cdot K=G$ and 
\item $\la a\ra\cap K=\{e\}$.
\end{itemize}
\end{lemma}

The proof given in class is like the one in Gallian or Judson.

\begin{theorem}
Any finite abelian group is isomorphic to a product of cyclic
groups each of which has prime-power order.
\end{theorem}

\begin{prob}(Wed Jan 29)
Let $G$ be a finite abelian group.  Prove that the following are equivalent
\begin{enumerate}
\item For every subgroup $H$ of $G$ there is a subgroup $K$ of $G$ with
$HK=G$ and $H\cap K=\{e\}$. 
\item Every  element of $G$ has square-free order.
\end{enumerate}
\end{prob}

Hint: Polya's Dictum: ``If you can't do a problem, then there is an
easier problem you can't do.  Find it.''

Lets call property (1) the Complementation Property for $G$
or CP for short.  Here are some easier problems:
\begin{enum}
\item Prove that $C_{p^2}$ fails to have CP.
\item Prove that $C_p\times C_p$ has CP.
\item Let $|G|$ and $|H|$ be
relatively prime.  Prove that $G\times H$ has CP iff both
$G$ and $H$ have CP.
\end{enum}


\end{document}
